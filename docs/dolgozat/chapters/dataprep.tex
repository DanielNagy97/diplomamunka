\Chapter{Adathalmaz előkészítése a tanításhoz}

A tanításhoz szükségünk van egy megfelelő adathalmazra. Képek generálásához legegyszerűbb esetben elegendő lehet képek rendezetlen halmaza is és a modellre bízhatjuk, hogy ismerje fel az egyes osztályok jellegzetességeit. Megfelelő regularizációs technikákkal igazán változatos képek generálására lesz képes a betanított modell.

Ha az adathalmaz rendelkezik osztályokkal is, úgy az osztály-címkéket is felhasználhatjuk a GAN hálózatunk tanításához. Így egy plusz bemenet segítségével könnyebben tudunk majd megfelelő képeket előállítani. (Ezt a technikát class-conditioningnak nevezik, de utána kell nézzek)

Egyes adathalmazokhoz igen részletes annotációkat is mellékelnek. A képeken megfigyelhető objektumokat határoló dobozokkal, vagy pixel szinten is jelölik. Így igen részletes információkat kínálnak a kép tartalmáról. Egy igazán hasznos annotációk lehetnek a természetes nyelvű leíró mondatok is, amelyekből általában többet is mellékelnek egy-egy képhez.

Ha kiválasztottuk a modellünkhöz megfelelő adathalmazt, akkor a probléma és a modell bonyolultságától függően elő kell készítenünk az adatokat a tanításhoz.

TODO: Data augmentation technikák összeszedése, demózás, kipróbálás az egyik modellen

\Section{Data augmentation - tanítás kevés adattal}

- karras2020training
- noguchi2019image