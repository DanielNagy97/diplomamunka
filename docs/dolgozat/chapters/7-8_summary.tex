\Chapter{Összegzés}

%TODO: Összegzés megírása, miután már összeállt a dolgozat!

Hasonló szerepe van, mint a bevezetésnek.
Itt már múltidőben lehet beszélni.
A szerző saját meglátása szerint kell összegezni és értékelni a dolgozat fontosabb eredményeit.
Meg lehet benne említeni, hogy mi az ami jobban, mi az ami kevésbé jobban sikerült a tervezettnél.
El lehet benne mondani, hogy milyen további tervek, fejlesztési lehetőségek vannak még a témával kapcsolatban.

\Chapter{Summary}

%TODO: Summary in english

The content of the previous chapter in english.