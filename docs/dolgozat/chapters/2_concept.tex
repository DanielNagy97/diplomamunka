\Chapter{Koncepció}

\Section{Az objektumfelismerés}
Az objektumfelismerés célja a számítógépes látás esetében, hogy egy gépi tanulásos modell egy vizsgált képen meghatározza a képen található objektumokat. Egyszerűbb esetben osztályozási feladatról beszélhetünk, amikor a bemeneti képeket egy-egy diszkrét címkével látja el a modell. A modell kimenete rendszerint az osztályokhoz tartozó valószínűségi értékek lesznek. A képek osztályozásához általánosságban konvolúciós rétegek segítségével nyerik ki a képekre vonatkozó információkat és egyúttal csökkentik a bemeneti dimenzószámot is. Kiemelném az Inception modellt \cite{szegedy2015going}, amely a Google kutatói által kifejlesztett mély konvolúciós hálózat. Fő céljuk egy olyan modell megalkotása volt, amely az ImageNet adathalmaz \cite{deng2009imagenet} 1000 darab osztályára megoldja az osztályozási feladatot. Az eredeti modell 2014-es megjelenése óta három további verziót is megélt (v2/v3 \cite{szegedy2016rethinking}, v4 \cite{szegedy2017inception}).
Ha a képen található objektum környezetéről feltételezzük, hogy nem hordoz számunka lényeges információkat, úgy a címkék használata elegendő lehet bizonyos feladatoknál. Viszont előfordulhat olyan alkalmazási környezet is, amikor a bemeneti képeken nem csupán egyetlen objektum található és mindegyikhez szeretnénk rendelni címkét. Ilyen esetekben a felismert objektumokat különféle annotációkkal láthatja el a modell. Az azonosított elemeket befoglaló dobozokkal \cite{redmon2016you} vagy pixel-szinten is jelölheti a modell, az utóbbi esetben szemantikus szegmentációról beszélhetünk \cite{long2015fully}.
Külön kiemelném a CLIP \textit{zero-shot} osztályozót \cite{radford2021learning}, amelyet az internetről összegyűjtött képeken és a hozzájuk tartozó szöveges leírásokon tanítottak be. A CLIP modell képes rövid leírásokat is adni a kép tartalmáról és az egyes mondatokat valószínűségi értékekkel látja el. Vagyis a különféle annotációk helyett természetesnyelvi leírásokat ajánl egy-egy képhez. A modell az inverz problémakör több megvalósításában is feltűnik majd amelyekre a következőkben kitérek.



\Section{Az objektumfelismerés inverz problémája}

Az objektumfelismerés inverz problémája alatt azt a feladatot értjük, amely során csupán a képekre vonatkozó információk állnak redelkezésünkre és ezen adatokat a gépi tanulásos modellnek a lehető legjobban kell reprezentálnia egy-egy kép formájában. 
Ehhez nem csupán a bemeneti, általában természetesnyelvi szöveget kell értelmeznie a modellnek, hanem a képek kigenerálásának módja is megoldandó probléma.


TODO: Különböző szerzők különféle megközelítéseinek bemutatása!

Egyes szerzők a minél valóságűbb eredményekre törekedtek, ehhez vagy Variational Autoencoder alapú architektúrát használtak \cite{ramesh2021zero}, vagy a Generative Adverserial Network alapokon nyugvó megoldásokat \cite{dong2021unsupervised, reed2016learning, xu2018attngan, zhang2017stackgan}.

\Section{A fejezet célja}

Ez a fejezet még nem a saját eredményekkel foglalkozik, hanem bemutatja, mi a problémakör, milyen módszerekkel, milyen eredményeket sikerült elérni eddig másoknak.


TODO: Problémakör bemutatása, különböző szerzők eredményeinek ismertetése


A hivatkozások jelentős része ehhez a fejezethez szokott kötődni.
(Egy hivatkozás például így néz ki \cite{coombs1987markup}.)
Ez szintén egy példa internetes hivatkozásra, a CSS szabvány kapcsán \cite{css}.
Itt lehet bemutatni a hasonló alkalmazásokat.

\Section{Tartalom és felépítés}

A fejezet tartalma témától függően változhat. Az alábbiakat attól függően különböző arányban tartalmazhatják.
\begin{itemize}
\item Irodalomkutatás. Amennyiben a dolgozat egy módszer kidolgozására, kifejlesztésére irányul, akkor itt lehet részletesen végignézni (módszertani vagy időrendi bontásban), hogy az eddigiekben milyen eredmények születtek a témakörben.
\item Technológia. Mivel jellemzően kutatásról vagy szoftverfejlesztésről van szó, ezért annak a jellemző elemeit, technikai részleteit itt kell bemutatni.
Ez tehát egy módszeres bevezetés ahhoz, hogy ha valaki nem jártas a témakörben, akkor tudja, hogy a dolgozat milyen aktuálisan elérhető eredményeket, eszközöket használt fel.
\item Piackutatás. Bizonyos témáknál új termék vagy szolgáltatás kifejlesztése a cél.
Ekkor érdemes annak alaposan utánanézni, hogy aktuálisan milyen eszközök érhetők el a piacon.
Ez szoftverek esetében a hasonló alkalmazások bemutatását, táblázatos formában történő összehasonlítását jelentheti.
Szerepelhetnek képek és észrevételek a viszonyításként bemutatott alkalmazásokhoz.
\item Követelmény specifikáció. Külön szakaszban érdemes részletesen kitérni az elkészítendő alkalmazással kapcsolatos követelményekre.
Ehhez tartozhatnak forgatókönyvek (\textit{scenario}-k).
A szemléletesség kedvéért lehet hozzájuk képernyőkép vázlatokat is készíteni, vagy a használati eseteket más módon szemléltetni.
\end{itemize}

\Section{Amit csak említés szintjén érdemes szerepeltetni}

Az olvasóról annyit feltételezhetünk, hogy programozásban valamilyen szinten járatos, és a matematikai alapfogalmakkal sem ebben a dolgozatban kell megismertetni.
A speciális eszközök, programozási nyelvek, matematikai módszerekk és jelölések persze jó, hogy ha említésre kerülnek, de nem kell nagyon belemenni a közismertnek tekinthető dolgokba.
