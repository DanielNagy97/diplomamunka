\pagestyle{empty}

\noindent \textbf{\Large CD melléklet tartalma}

\vskip 1cm

\noindent A dolgozathoz mellékelt lemezen egy \texttt{Dolgozat} nevű jegyzékben a következő fájlok találhatóak.

\begin{itemize}
\item A dolgozat \LaTeX\ forráskódja.
\item A dolgozat PDF formátumban (\texttt{dolgozat.pdf}).
\item A magyar és angol nyelvű összefoglaló \LaTeX\ és PDF formátumban \\ (\texttt{osszegzes.tex}, \texttt{osszegzes.pdf}, \texttt{summary.tex}, \texttt{summary.pdf}).
\end{itemize}

% Az \texttt{example-programs} nevű jegyzékben találhatóak a GAN hálózatokhoz készített \texttt{catgan} és \texttt{biggergantest} Python csomagok.

A \texttt{notebooks} jegyzékben találhatóak a kutatások során készített Jupyter munkafüzetek, a hozzájuk tartozó adatok, ábrák és maga a \texttt{obinv} csomag is.
