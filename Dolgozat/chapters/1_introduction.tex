\Chapter{Bevezetés}

A képfeldolgozás, mint alkalmazási terület, a mély tanulási módszerek (\textit{Deep Learning}) egy kedvelt, és igen látványos része.
A mesterséges neurális hálózatokra vonatkozó kutatások eredményei, és az aktuálisan elérhető számítási kapacitás egyre általánosabb problémák megoldását teszi lehetővé.
Egy ilyen, közel sem triviális probléma az, hogy hogyan lehet színes képeket generálni az azokhoz rendelkezésre álló metainformációk (például kulcsszavak, szöveges leírás) alapján.
A dolgozat ezt a problémakört hivatott körüljárni, arra megoldást találni és a kapott eredményeket bemutatni.

A dolgozat elején röviden áttekintésre kerül az objektum felismerés problémaköre, mint a direkt feladat. Ez azért fontos, mert a továbbiakban ennek inverzére keressük a megoldást, vagyis szeretnénk generálni olyan képeket, amelyen bizonyos objektumok vannak, illetve teljesülnek a képre valamilyen, alapvetően szöveges formában megadott kritériumok. A problémakör nem tekinthető újnak, ezért bemutatásra kerülnek a szakirodalomban fellelhető jelentősebb kutatások, módszerek és azok eredményei.

A feladat megoldásához elsősorban a GAN hálózatok használatára esett a választás, melynek implementálását a Tensorflow és a Keras függvénykönyvtárak segítik.

Mivel gépi tanulási problémáról van szó, ezért elengedhetetlen, hogy megfelelő tanítóminták álljanak rendelkezésre. Ehhez számos mintaadathalmaz (\textit{dataset}) elérhető, melyeket szintén a 2. fejezet taglal. Ezt követi néhány futtatókörnyezettel kapcsolatos vizsgálat, mivel a gépi tanulásnál használt algoritmusok (a mély tanulási módszerek pedig különösen) nagyobb erőforrásokat igényelnek számítási kapacitás és tárigény tekintetében egyaránt.

A képgeneráláshoz a bemenetet valamilyen strukturált vagy strukturálatlan szöveges jellegű adat jelenti. A harmadik fejezetben részletezésre kerül ezek feldolgozási módja, típusaik és konkrét példák a megjelenési módjukra.

A megoldandó probléma szempontjából a képek generálási módjának kiemelt szerepe van. Emiatt a GAN hálózat modellje, a szükséges hibafüggvények és optimalizálási módszerek az elméleti háttérrel és az implementációs móddal együtt kifejtésre kerül majd.

Külön nehézséget jelent a generált képek jóságának a megítélése. Az \textit{Inception Score} és a \textit{Fréchet Inception Distance} lesz majd alkalmas arra, hogy ezt megfelelően számszerűsíteni tudja. Érdekes problémaként említésre kerül a \textit{mode-collapse} jelenség. Ennek, és a hasonló tanítási problémák lehetséges megoldásairól regularizációs módszerek formájában esik majd szó.

A mély tanulási módszerek kapcsán találkozhatunk a GAN modell különféle változataival, kibővítési lehetőségeivel. A 4. fejezetben ezekre találhatunk példákat, azok architekturális leírásával és a hálók segítségével kapott eredmények értékelésével együtt.

Az 5. fejezetben egy olyan megközelítésről olvashatunk, melyben a Generátort adottnak tekintjük, és a képgenerálási problémát a megfelelő zajvektor keresésére vezetjük vissza. Ehhez a szükséges metrikák, a gradiens alapú keresési módszer egyaránt kifejtésre kerül.

A dolgozat végén konkrét futtatási példákat láthatunk a hozzájuk tartozó paraméterezéssel, a kapott eredménnyel és a tanítási folyamat áttekintésével együtt.
