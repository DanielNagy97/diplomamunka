\Chapter{Összegzés}

A dolgozat megírása során elvégzett kutatások megerősítették, hogy egy igen komoly, szerteágazó és aktívan kutatott területről van szó. A fellelhető, a szűkebb problémakör szempontjából releváns szakirodalom összegyűjtése, rendszerezése már önmagában is kihívást jelentett. A tudományos cikkekben közölt rengeteg eshetőség közül ki kellett tudni választani a megfelelőket.

A vizsgálatokhoz a technológiák adottnak tekinthetők voltak, mivel gépi tanulás problémakörben a Python programozási nyelv, a Jupyter munkafüzetek, a TensorFlow és a Keras függvénykönyvtárak használata a leginkább elterjedt. Külön figyelmet kellett fordítani azonban a futtatókörnyezet megválasztására, mivel már az első vizsgálatok kapcsán kiderült, hogy azokat nem célszerű lokálisan futtatni.

A képek generálásához a GAN modell tünt és bizonyult megfelelőnek. Ennek különféle változatai, paraméterezései, és az általuk kapott eredmények bemutatásra kerültek.

A dolgozatban láthattunk példákat a \textit{mode-collapse} jelenségre, annak különféle eseteire. Ezt saját mérésekkel is sikerült elemezni. A kapott hibagörbéken jól kivehető az a pont, amely után a háló gyakorlatilag használhatatlanná válik.

A 4. fejezetben különféle hálózati topológiát, réteg összeállítást, és a velük végzett mérések eredményeit be sikerült mutatni.

Az 5. fejezetben egy olyan módszer szerepelt, amely a képgenerálást a hibavektorok terében való kereséssel oldja meg. Ehhez implementálásra került egy gradiens alapú módszer, melyet momentummal is ki sikerült egészíteni.

A dolgozathoz elkészült egy \texttt{obinv} nevű Python csomag, amely összegzi, és kényelmesen elérhetővé teszi a kutatások eredményei alapján kiválasztott képgenerálási módokat.

A probléma általánosságban közel sem tekinthető megoldottnak. Rengeteg további lehetőség van a hálók kialakítására, paraméterek és metrikák megválasztására vonatkozóan. Ezek már további kutatások tárgyát képezik.
